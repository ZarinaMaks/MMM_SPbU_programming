\documentclass[a4paper,12pt]{article}
\usepackage{cmap}
\usepackage {listings}
\usepackage{verbatim}
\usepackage {misccorr}
\usepackage{graphicx}
\usepackage[T2A]{fontenc}
\usepackage[utf8]{inputenc}
\usepackage[english,russian]{babel}
\usepackage{amsfonts}
\author{Зарина}
\date{декабрь 2019}
\usepackage{indentfirst}
\begin{document}
13. Жадные алгоритмы: матроиды, теорема Радо-Эдмондса, жадный алгоритм поиска базы минимального веса в матроиде
\section{Матроид} 
Матроид - пара $<X,I>$, где $X$  — конечное множество, называемое носителем матроида, а $I$ — некоторое множество подмножеств $X$, называемое семейством независимых множеств: $I \subset 2^X$\\

При этом должны выполняться следующие условия:\\
1) $\varnothing \in I$\\
2) если $A \in I$ и $B \subset A$, то $B \in I$\\
3) если $A,B \in I$ и $|A|>|B|$, то $ \exists x \in A \setminus B \mid B\cup x \in I$\\

База матроида - максимальное подключение элементов множества, входящее в матроид(или же "множество максимальной мощности, входящее в матроид");\\


\textbf{Взвешенный матроид:}

$A,B \in I$ и $A\cap B= \varnothing $ => $W(A\cup B)=W(A)+W(B)$\\

\textbf{Примеры матроидов:}\\

1)	Разноцветный матроид\\
Пусть $X$ — множество элементов, каждый из которых раскрашен в некоторый цвет. Множество $A \in I$, если все элементы множества A разного цвета. Тогда $M=<X,I>$ называется разноцветным матроидом.\\

2)	Графовый матроид\\
Пусть $G=<V,E>$ — неориентированный граф. Тогда $M=<E,I>$, где $I$ состоит из всех ацикличных множеств ребер (то есть являющихся лесами), называют графовым (графическим) матроидом.\\

3)	Матричный матроид\\
Пусть $V$ — векторное пространство над телом $F$, пусть набор векторов $V_{i}=f_{v1}, …, f_{vng}$ из пространства $V$ является носителем $X$. Элементами независимого множества $I$ данного матроида являются множества линейно независимых векторов из набора $v_{1},..., v_{n}$. Тогда $M=<V_{i},I>$, называется матричным матроидом\\

4)	Бинарный матроид\\
Матроид M представим над полем F, если он изоморфен некоторому векторному матроиду над этим полем.\\
Бинарный матроид — матроид, представимый над полем целых чисел по модулю 2.

\section{Теорема Радо-Эдмондса} 
На носителе матроида $M=(X,I)$ задана весовая функция $W:X->R$.\\
$A\in I$ - множество минимального веса среди всех множеств мощности $k$.\\
Возьмем $x$, т.ч.  $A\cup x \subset I$, $x \not \subset A$, $W(x)-min$\\

Тогда  $A\cup x$ - множество минимального веса из множеств мощности $(k+1)$, входящих в $I$.\\

\textbf{Доказательство:}\\

Рассмотрим $B\in I$ — множество минимального веса среди независимых подмножеств $X$ мощности $(k+1)$.\\

Из определения матроида: $\exists y \in B\setminus A \mid A \cup y \in I$.\\

Тогда верны два неравенства:\\
$W(A\cup y)=W(A)+W(y) \geq W(B)\Rightarrow W(A) \geq W(B)-W(y)$;\\
$W(B \setminus y)=W(B)-W(y) \geq W(A)$.\\

Заметим, что величина $W(A)$ с двух сторон ограничивает величину $W(B)−W(y)$. Значит, эти величины равны: $W(A)=W(B)-W(y) \Rightarrow W(A)+W(y)=W(B)$.\\

Следовательно, $W(A\cup y)=W(A)+W(y)=W(B)$.\\

Таким образом получаем, что если объединить множество $A \subset x$— минимальным из таких, что $A\cup x \in I$, — то получим множество минимального веса среди независимых подмножеств $X$ мощности $(k+1)$.

\section{Поиск базы минимального веса} 
Поиск базы минимального веса является следствием из теоремы Радо-Эдмондса:\\

Для любого $A\subset X$ выполнено: $W(A)=\sum\limits_{x\in A} W(x)$. Тогда база минимального веса матроида M ищется жадно.\\

\textbf{Алгоритм поиска базы:}\\
1) сортируем X по весу: $sort (X,W)$\\
 2) $B= \varnothing $\\
 3) $for \: i=1..n-1$\\
 $if (B\cup X[i]) \subset I$\\
 $B=B\cup {X[i]}$\\
 Мощность увеличивается, пока мы не дойдем до базы.\\
 
 \textbf{Асимптотика алгоритма:} $O(nlogn+nm)$:\\
 На сортировку элементов из $X$ по возрастанию весов уходит $O(nlogn)$. После чего, построение базы выполняется $O(n)$ шагов цикла, каждый из которых работает $O(m)$ времени.\\
 Замечание: если считать, что проверка множества на независимость происходит за $O(1)$, асимптотика алгоритма будет $O(nlogn)$
 
 \section {Задачи}
\\Рассмотрим несколько задач: \\

\textbf{Задача 1}\\
Известна стоимость построения дороги между городами. Нужно построить нецикличную систему дорог, затратив минимальные ресурсы:\\
1)Упорядочим по стоимости все дороги\\
2)Применим алгоритм Краскала - эффективный алгоритм построения минимального остовного дерева взвешенного связного неориентированного графа. \\
В начале текущее множество рёбер устанавливается пустым. Затем, пока это возможно, проводится следующая операция: из всех рёбер, добавление которых к уже имеющемуся множеству не вызовет появление в нём цикла, выбирается ребро минимального веса и добавляется к уже имеющемуся множеству. Когда таких рёбер больше нет, алгоритм завершён. Подграф данного графа, содержащий все его вершины и найденное множество рёбер, является его остовным деревом минимального веса.


\textbf{Задача 2}\\
Имеются n задач и дедлайны к ним. Имеются функции потерь(=неполученная прибыль). =>
\begin{tabular}{ccc}
1 & \textbf{t1} & C1 \\
2 & \textbf{t2} & C2 \\
3 & \textbf{t3}& C3 \\
\end{tabular}\\
1) Отсортируем сначала по дедлайну\\
2) В зависимости от расположения дедлайнов, оптимизируем\\
 
 
\end{document}
