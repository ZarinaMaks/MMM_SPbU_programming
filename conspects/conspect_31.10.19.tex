\documentclass[a4paper,12pt]{article}
\usepackage{cmap}
\usepackage{amssymb}
\usepackage[T2A]{fontenc}
\usepackage[utf8]{inputenc}
\usepackage[english,russian]{babel}
\author{Зарина}
\title{Матроиды}
\date{31 окрября 2019}
\begin{document}
\section {Несколько задач}
\textbf{Задача 1}\\
Известна стоимость построения дороги между городами. Нужно построить нецикличную систему дорог, затратив минимальные ресурсы:\\
1)Упорядочим по стоимости все дороги\\
2)Применим алгоритм Краскала\\
\textbf{Задача 2}\\
Имеются n задач и дедлайны к ним. Имеются функции потерь(=неполученная прибыль). =>
\begin{tabular}{ccc}
1 & \textbf{t1} & C1 \\
2 & \textbf{t2} & C2 \\
3 & \textbf{t3}& C3 \\
\end{tabular}\\
1) Отсортируем сначала по дедлайну\\
2) В зависимости от расположения дедлайнов, оптимизируем\\
\section{МАТРОИД - $<X,I>$}
-система независимых множеств\\
Множество $I$ содержится в $2^X$ (в наборе подмножеств множества X)\\ 
База матроида - максимальное подключение элементов множества, входящее в матроид(множество максимальной мощности, входящее в матроид);
\section {Взвешенный матроид}
$A,B \in I$ и $A\cap B= \varnothing $ => $W(A\cup B)=W(A)+W(B)$
\section{Теорема Радо-Эдмондса}
$M=(X,I)$; $W:X->R$ - задана весовая функция\\
$A\in I$ - множество минимального веса среди всез множеств мощности k\\
Возьмем X, т.ч.  $A\cup X \in I$, $X \notin A$, $W(X)-min$\\
Тогда  $A\cup X$ - множество минимального веса из множеств мощности (k+1), входящих в I.\\
 \textbf{Следствие:} жадный алгоритм поиска базы минимального веса:\\
 1) $sort (X,W)$ - сортируем X по весу\\
 2) $B= \varnothing $\\
 3) $for \: i=1..n-1$\\
 $if (B\cup X[i]) \in I$\\
 $B=B\cup {X[i]}$\\
 Мощность увеличивается, пока мы не дойдем до базы.\\
 $O(nlogn+nm)$ - асимптотика алгоритма
\end{document}