\documentclass[a4paper,12pt]{article}
\usepackage{cmap}
\usepackage[T2A]{fontenc}
\usepackage[utf8]{inputenc}
\usepackage[english,russian]{babel}
\author{Зарина}
\title{Задачи о "Рюкзаке"}
\date{6 ноября 2019}
\begin{document}
\section {Задача о "рюкзаке"}
Непрерывная задача о рюкзаке рассматривает неделимые объекты (стоимость пропорциональна массе)

\section{Жадный алгоритм}
1) Выберем сначала макс.уд.стоимость\\
2) Возьмем этот предмет столько, сколько сможем\\
=> Существует класс жадных алгоритмов, не решающих задачу оптимально, но приводящих к оптимальному решению - градиентный метод.\\
Например, поиск глобального минимума через локальные минимумы.
\section {Градиентный метод}
\textbf{1)Метод бисекции.}\\
-простейший численный метод для решения нелинейных уравнений вида f(x)=0. Предполагается только непрерывность функции f(x). Поиск основывается на теореме о промежуточных значениях:
Для A<0, B>0 на отрезке [A;B] существует точка, пересекающая ось абсцисс.\\
\textbf{2)Метод Ньютона.}\\
-если $x_{n}$ — некоторое приближение к корню  уравнения f(x)=0, то следующее приближение определяется как корень касательной к функции f(x), проведенной в точке $x_{n}$ .\\
\textbf{3)} Значение градиента в точке - вектор, направление которого задает значение производной в точке f(x,y,z).
С помощью этого метода обучается нейронная сеть.\\
\section{Динамическое программирование}
-довольно близко к жадному алгоритму\\
Примеры:\\
\textbf{1) Числа Фибоначчи}\\
Можно воспользоваться формулой Бине.\\
Пусть есть массив a[n+1] ={-1,-1,-1...-1}\\
$f(n):\\
if (a[n]==-1)\\
a[n]=f(n-1)+f(n-2);\\
return\:a[n]\\$
Мы не делаем лишних вычислений, а заполняем промежуточные результаты и пользуемся ими:\\
1) Сформулировали задачу в рекуррентной форме\\
2) Последовательно заполнили массив обратным ходом\\
\textbf{2) Треугольник Паскаля}\\

\end{document}